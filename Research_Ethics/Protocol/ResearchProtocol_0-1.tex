\documentclass{article}


\title{Study Protocol}
\author{Ping-Chen Tsai  \thanks{Supervisor: Dr. Lanaky Heba}}
\date{07 January 2021}
\usepackage[utf8]{inputenc}
\usepackage{xcolor}
\usepackage{xurl}
\usepackage{geometry}
\usepackage{microtype} %solve badbox
\usepackage[style=ieee, backend=biber, url=false, doi=false, isbn=false]{biblatex}
\usepackage{float}
\usepackage{comment}
\usepackage{enumitem} % for bullet points

\usepackage{color}   %May be necessary if you want to color links
\usepackage{hyperref}
\hypersetup{
    colorlinks,
    citecolor=blue,
    filecolor=black,
    linkcolor=black,
    urlcolor=black
}

\usepackage{graphicx}
\graphicspath{ {./image/} }
\addbibresource{ResearchProtocol.bib}

\setlength{\parskip}{1em}
\geometry{a4paper,scale=0.8}

\begin{document}

\maketitle

\tableofcontents
\pagebreak


\section{Background}
\begin{comment}
Aim: To place the study in the context of available evidence.
The background should be supported by appropriate references to published literature on the area of interest:
•	A thorough literature review of relevant studies and analysis, new research should build on formal review of prior evidence.
•	A brief description of the proposed study.
•	A description of the population to be studied.
•	relevant data from previous clinical trials such as efficacy, safety, tolerability

It should be written so it is easy to read and understand by someone with a basic sense of the topic who may not necessarily be an expert in the area. Some explanation of terms and concepts is likely to be beneficial. 
\end{comment}

Falls is one of the major causes to serious injuries in older people. About one-third of the frail adults age over 65 will experience at least one fall a year when living in their own homes, and the fall rate is three times higher in older people who are residents of care homes \cite{NHSFalls_2018, Robertson_2013}. Identifying and reducing risk of falls in the elderly hence have been extensively studied to carry out appropriate therapeutic interventions for the fallers. Risk assessment tools are used for falls screening and falls management. A wide range of fall risk assessment tools are used to identify fall risks of older people who are dwelling in the community \cite{PreventionofFalls2011} or in care homes \cite{Jung_2014, Kikkert_2017}. 

Ageing and mental disorders lead to gait and cognitive impairments. Correlation between cognitive impairments and fall risks have been identified in existing studies. Both literature evidence was provided and trials of experiments were conducted to in the studies to prove that older people with cognitive impairment are entitled to higher fall risk \cite{Yogev_Seligmann_2007,Martin_2012,Laurence_2017, Zhang_2019}. These studies mainly focused on examining damaged or declined cognitive functions impacts on gait in older people, and their relations to falls. Since falling in older people happened mostly when gait and balance control is poor, risk factors related to gait variables were investigated \cite{Zhang_2019, Mirelman_2012, Borowicz_2016}. Results from the studies ascertained that aspects of execution function including attention, processing speed and working memory in the cognitive domain are essential to control gait and postural tasks for human body. An appropriate intervention to improve cognitive function in older people therefore becomes a possibility to reduce the fall risk \cite{Barban_2017}. 



\section{Rationale}
% motivation: why we need to conduct the study? Why is the study reasonable?
\begin{comment}
Aim: To explain why the research questions/aim(s) being addressed are important and why closely related questions are not being covered. 
This should include:
•	A clear explanation of the research question/aim(s) and the justification of the study i.e. why the question is worth asking and, through consultation with public and patient groups, why this is worthwhile to participants or wider service delivery.
•	A contextual framing of the research question/aim(s) in relation to relevant policy and historical and/or literature bases.
\end{comment}

Falls in older people are major health care issue worldwide. An evidence-based review study suggested that over 50\% of potential fall incidents happen on older people will be avoided if prevention interventions take place rigorously \cite{Kannus_2005}. So far, many studies have developed fall prevention guidelines that recommend interventions including physical activity, medical alleviation and care facilitation for older people with higher fall risks \cite{Jung_2014, PreventionofFalls2011,Church_2015}. Although cognitive decline is a target risk factor of falling for older adults, the guidelines provide fall risk prevention recommendations related to cognitive decline in a indirect and sparse manner. In fact, recent trial of studies provided evidence of beneficial effects on fall risk reduction using computer-based cognitive training, such as motor training \cite{van_het_Reve_2014, Barban_2017}. More specifically, older people improved gait to control postural sway as a result of enhancement in attention and cognitive processing speed by the cognitive training. As the studies proposed were not explicitly conclusive, result evidence from the existing studies encourages future work to be carried out. 

The last decade has seen the rapid ascent of non-invasive brain-computer-interface(BCI) technology. BCI provides communication between human brain and an external device. One randomized controlled study has showed the safety, usability, acceptability and efficacy of BCI intervention in older people \cite{Lee_2013}. Evidence has supported that such training could improve cognitive performance of older adults with mild cognitive impairments by improving their EEG activities \cite{Marlats_2020}. Executive function \cite{Mond_jar_2016}, processing speed \cite{Doppelmayr_2011} and attention \cite{Arvaneh_2019} enhancement using EEG-based BCI neuralfeedback training was proved to be effective. \citeauthor{Mond_jar_2016} proposed  that by implementing certain mechanism in action videogames that applies EEG-based BCI, the game could potentially improve ones' executive skills. \citeauthor{Doppelmayr_2011} conducted a series of 30 sensorimotor rhythm(SMR) training sessions on healthy individuals and found that the SMR training resulted in faster reactions and better visuospatial abilities. In \citeauthor{Arvaneh_2019}'s work,a series of letter(stimuli)were shown on a monitor for the healthy young participants to raise attention by volitionally choose the target letter. The BCI then identified the target EEG component(the P300 stimuli) in the brain and provided frequent feedback to the participant when the letter was correctly chosen. After a short-time-scale training, the experimental group was found to be less distracted compared to the non-trained group which indicated improvement in their attention. 

Non-invasive BCI intervention requires slighter and less-expensive implementation compare to conventional fall-prevention programmes such as physical therapy and reconstruction of living environment. A widely used BCI therapy for improving cognitive function is electroencephalogram (EEG) based neurofeedback(NF) training. To date, emerging studies on the use of Brain-Computer-Interface(BCI) for cognitive training supported that BCI could strengthen neural plasticity by neurofeedback (NF) training. In particular, neurofeedback training applies BCI to change cognitive process and provide real-time feedbacks in either visual or acoustic form. Considering fall risk reduction, present study will bridge the gap between fall risk and cognitive decline using brain-computer-interface technology. The aim of the study is to investigate the performance of BCI approach on reducing fall risks in older adults.

\section{Theoretical Framework}
\begin{comment}
To describe the theoretical framework for the study.
•	A clear explanation of the proposed approach and why it is suitable to address the gaps outlined in the BACKGROUND section. 
•	Briefly outline a system of concepts, from published literature, that frames your study.
•	Can be presented either visually or textually. 
\end{comment}
EEG based NF is a supportive training where the participant modulates his/her EEG activity based on real-time visual or auditory feedback from real world. Since EEG signals has been proved to expose one's intent and cognitive performance, expected EEG patterns transmitted from a healthy elderly could be the reference for a NF training. With repetitive volitional practice, reinforcement and sensory feedback,the participants will learn to better control neural activity that imposes positive effects on cognitive functions and gait or balance control \cite{Miladinovic_2020}. 

The application of NF training deems to simulate or inhibit the target frequency bands in EEG to prompt cognitive changes(Lower and upper alpha: 7-12 Hz; beta: 15-20 Hz; gamma: 30-80 Hz; delta: 0-4 Hz; theta: 4-7 Hz). For example, upper alpha wave is closely related to attentional process, with lower alpha wave has association with semantic memory.Theta wave appears to indicate the activities in short-term memory(both working memory and episodic memory)\cite{Lecomte_2011}. The complexity between EEG signals and cognitive function means that it is not easy to set a target training protocol on frequency bands simulation or inhibition. Despite the uncertain nature of the association, findings from previous studies\cite{Asheri_2018, Doppelmayr_2011} may be enough to suggest the hypothesis of correlation between EEG activity and cognitive performance . (\textcolor{red}{add some examples})


While NF training tasks are usually monotonous and repetitive, training tasks in the form of gaming are usually preferred. Game-like features in cognitive training can provide an intuitive rule to engage the participant which gives them self-efficacy \cite{McGonigal_2011}. This is to prevent the older participants from sleeping or bored in the training session. As a result, the training task is better to be entertaining enough for the participant to feel motivated and rewarded. 

Taken together, the neurofeedback method in the present study will game-based. The user will be able to immerse in the gaming environment for their cognitive improvement training sessions. 

\section{Research Aim and Objectives}
\begin{description}[font=$\bullet$~\normalfont\textbf]
\item [Research Question:] Can neuralfeedback training by BCI reduce risk of falls in older people by improving their cognitive function?
\item [Present hypothesis:] Neuralfeedback training by BCI can reduce risk of falls in older people by improving cognitive function. 
\end{description}

\section{Outcome}
\begin{comment}
Aim: To outline potential broad outcomes for the study which will reflect the research question aim(s).
\end{comment}


\section{STUDY DESIGN and METHODS of DATA COLLECTION AND DATA ANALYIS}
\begin{comment}
Aim: To describe the study design. To clearly describe the data collection methods and outline the roles involved in data collection. To clearly describe the data analysis methods.
\end{comment}

To evaluate the effects of neurofeedback method, we plan to assess a variety of health features using clinical screening Tools. Gait and balance will be measured by Berg Balance Test. Frailty will be measured by Rockwood Frailty. Fall risk will be assessed by FRAT. Performance of daily living activity will be measured by Barthel. 


\printbibliography


\end{document}

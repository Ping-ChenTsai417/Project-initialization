\documentclass{article}


\title{Study Protocol}
\author{Ping-Chen Tsai  \thanks{Supervisor: Dr. Lanaky Heba}}
\date{03 March 2021}
\usepackage[utf8]{inputenc}
\usepackage{xcolor}
\usepackage{xurl}
\usepackage{geometry}
\usepackage{microtype} %solve badbox
\usepackage[style=ieee, backend=biber, url=false, doi=false, isbn=false]{biblatex}
\usepackage{float}
\usepackage{comment}
\usepackage{enumitem} % for bullet points

\usepackage{color}   %May be necessary if you want to color links
\usepackage{hyperref}
\hypersetup{
    colorlinks,
    citecolor=blue,
    filecolor=black,
    linkcolor=black,
    urlcolor=black
}

\usepackage{graphicx}
\graphicspath{ {./image/} }
\addbibresource{ResearchProtocol.bib}

\setlength{\parskip}{1em}
\geometry{a4paper,scale=0.8}

\begin{document}

\maketitle

\tableofcontents
\pagebreak


\section{Background}
\begin{comment}
Aim: To place the study in the context of available evidence.
The background should be supported by appropriate references to published literature on the area of interest:
•	A thorough literature review of relevant studies and analysis, new research should build on formal review of prior evidence.
•	A brief description of the proposed study.
•	A description of the population to be studied.
•	relevant data from previous clinical trials such as efficacy, safety, tolerability

It should be written so it is easy to read and understand by someone with a basic sense of the topic who may not necessarily be an expert in the area. Some explanation of terms and concepts is likely to be beneficial. 
\end{comment}

Falls is one of the major causes to serious injuries in older people. About one-third of the frail adults age over 65 will experience at least one fall a year when living in their own homes, and the fall rate is three times higher in older people who are residents of care homes \cite{NHSFalls_2018, Robertson_2013}. Identifying and reducing risk of falls in the elderly hence have been extensively studied to carry out appropriate therapeutic interventions for the fallers. Risk assessment tools are used for falls screening and falls management. A wide range of fall risk assessment tools are used to identify fall risks of older people who are dwelling in the community \cite{PreventionofFalls2011} or in care homes \cite{Jung_2014, Kikkert_2017}. 

Ageing and mental disorders lead to gait and cognitive impairments. Correlation between cognitive impairments and fall risks have been identified in existing studies. Both literature evidence was provided and trials of experiments were conducted to in the studies to prove that older people with cognitive impairment are entitled to higher fall risk \cite{Yogev_Seligmann_2007,Martin_2012,Laurence_2017, Zhang_2019}. These studies mainly focused on examining damaged or declined cognitive functions impacts on gait in older people, and their relations to falls. Since falling in older people happened mostly when gait and balance control is poor, risk factors related to gait variables were investigated \cite{Zhang_2019, Mirelman_2012, Borowicz_2016}. Results from the studies ascertained that aspects of execution function including attention, processing speed and working memory in the cognitive domain are essential to control gait and postural tasks for human body. An appropriate intervention to improve cognitive function in older people therefore becomes a possibility to reduce the fall risk \cite{Barban_2017}. 



\section{Rationale}
% motivation: why we need to conduct the study? Why is the study reasonable?
\begin{comment}
Aim: To explain why the research questions/aim(s) being addressed are important and why closely related questions are not being covered. 
This should include:
•	A clear explanation of the research question/aim(s) and the justification of the study i.e. why the question is worth asking and, through consultation with public and patient groups, why this is worthwhile to participants or wider service delivery.
•	A contextual framing of the research question/aim(s) in relation to relevant policy and historical and/or literature bases.
\end{comment}

Falls in older people are major health care issue worldwide. An evidence-based review study suggested that over 50\% of potential fall incidents happen on older people will be avoided if prevention interventions take place rigorously \cite{Kannus_2005}. So far, many studies have developed fall prevention guidelines that recommend interventions including physical activity, medical alleviation and care facilitation for older people with higher fall risks \cite{Jung_2014, PreventionofFalls2011,Church_2015}. Although cognitive decline is a target risk factor of falling for older adults, the guidelines provide fall risk prevention recommendations related to cognitive decline in a indirect and sparse manner. The repetitive physical exercises could become a mental burden for the frailty and discourage the elderly from accepting interventions. In fact, recent trial of studies provided evidence of beneficial effects on fall risk reduction using computer-based cognitive training, such as motor training \cite{van_het_Reve_2014, Barban_2017}. More specifically, older people improved gait to control postural sway as a result of enhancement in attention and cognitive processing speed by the cognitive training. As the studies proposed were not explicitly conclusive, result evidence from the existing studies encourages future work to be carried out. 

The last decade has seen the rapid ascent of non-invasive brain-computer-interface(BCI) technology. BCI provides communication between human brain and an external device. One randomized controlled study has showed the safety, usability, acceptability and efficacy of BCI intervention in older people \cite{Lee_2013}. Evidence has supported that such training could improve cognitive performance of older adults with mild cognitive impairments by improving their EEG activities \cite{Marlats_2020}. Executive function \cite{Mond_jar_2016}, processing speed \cite{Doppelmayr_2011} and attention \cite{Arvaneh_2019} enhancement using EEG-based BCI neuralfeedback training was proved to be effective. \citeauthor{Mond_jar_2016} proposed  that by implementing certain mechanism in action videogames that applies EEG-based BCI, the game could potentially improve ones' executive skills. \citeauthor{Doppelmayr_2011} conducted a series of 30 sensorimotor rhythm(SMR) training sessions on healthy individuals and found that the SMR training resulted in faster reactions and better visuospatial abilities. In \citeauthor{Arvaneh_2019}'s work,a series of letter(stimuli)were shown on a monitor for the healthy young participants to raise attention by volitionally choose the target letter. The BCI then identified the target EEG component(the P300 stimuli) in the brain and provided frequent feedback to the participant when the letter was correctly chosen. After a short-time-scale training, the experimental group was found to be less distracted compared to the non-trained group which indicated improvement in their attention. 

Non-invasive BCI intervention requires slighter and less-expensive implementation compare to conventional fall-prevention programmes such as physical therapy and reconstruction of living environment. A widely used BCI therapy for improving cognitive function is electroencephalogram (EEG) based neurofeedback(NF) training. To date, emerging studies on the use of Brain-Computer-Interface(BCI) for cognitive training supported that BCI could strengthen neural plasticity by neurofeedback (NF) training. In particular, neurofeedback training applies BCI to change cognitive process and provide real-time feedbacks in either visual or acoustic form. Considering fall risk reduction, present study will bridge the gap between fall risk and cognitive decline using brain-computer-interface technology. The aim of the study is to investigate the performance of BCI approach on reducing fall risks in older adults.

\section{Theoretical Framework}
\begin{comment}
To describe the theoretical framework for the study.
•	A clear explanation of the proposed approach and why it is suitable to address the gaps outlined in the BACKGROUND section. 
•	Briefly outline a system of concepts, from published literature, that frames your study.
•	Can be presented either visually or textually. 
\end{comment}
EEG based NF is a supportive training where the participant modulates his/her EEG activity based on real-time visual or auditory feedback from real world. Since EEG signals has been proved to expose one's intent and cognitive performance, expected EEG patterns transmitted from a healthy elderly could be the reference for a NF training. With repetitive volitional practice, reinforcement and sensory feedback,the participants will learn to better control neural activity that imposes positive effects on cognitive functions and gait or balance control \cite{Miladinovic_2020}. 

The application of NF training deems to simulate or inhibit the target frequency bands in EEG to prompt cognitive changes(Lower and upper alpha: 7-12 Hz; beta: 15-20 Hz; gamma: 30-80 Hz; delta: 0-4 Hz; theta: 4-7 Hz). For example, upper alpha wave is closely related to attentional process, with lower alpha wave has association with semantic memory.Theta wave appears to indicate the activities in short-term memory(both working memory and episodic memory)\cite{Lecomte_2011}. The complexity between EEG signals and cognitive function means that it is not easy to set a target training protocol on frequency bands simulation or inhibition. Despite the uncertain nature of the association, findings from previous studies may be enough to suggest the hypothesis of correlation between EEG activity and cognitive performance. Concentration pattern was found in the brain EEG signals when playing a brain-computer interface maze game, where the game has potential to improve the disease attention deficit hyperactivity disorder(ADHD) \cite{Asheri_2018}. One of the EEG component, sensorimotor rhythm (SMR), was be trained on ageing population and the result concluded that SMR training is a positive tool that favours cognitive ageing especially in working memory aspect \cite{Campos_da_Paz_2018}. In addition, increase in amplitude of SMR could lead to faster reactions and good visuospatial abilities\cite{Doppelmayr_2011}. 

NF training reinforces the neural activation in the same brain region through repeated practice. While NF training tasks are usually monotonous and repetitive, training tasks in the form of gaming are usually preferred. Game-like features in cognitive training can provide an intuitive rule to engage the participant which gives them self-efficacy \cite{McGonigal_2011}. This is to prevent the older participants from sleeping or bored in the training session. As a result, the training task is better to be entertaining enough for the participant to feel motivated and rewarded. 


Taken together, the neurofeedback method in the present study will be game-based. The user will be able to immerse in the gaming environment for their cognitive improvement training sessions. 

\section{Research Question/Aim}
\begin{description}[font=$\bullet$~\normalfont\textbf]
\item [Research Question:] Can neuralfeedback training by BCI reduce risk of falls in older people by improving their cognitive function?
\item [Present hypothesis:] Neuralfeedback training by BCI can reduce risk of falls in older people by improving cognitive function. 
\end{description}

The research aims to explore the efficacy of neuralfeedback training using BCI to reduce fall risks in older people.

\subsection{Objectives}
\begin{description}[font=$\bullet$~\normalfont\textbf]
\item To conduct a experimental neuralfeedback training using BCI on older adults with mild cognitive impairments who are at higher risk of fall.
\item  To analysis the neurofeedback training effects of cognitive function improvement and assess the fall risks differences before and after training.
\end{description}

\subsection{Outcome}
\begin{comment}
Aim: To outline potential broad outcomes for the study which will reflect the research question aim(s).
\end{comment}
\begin{description}[font=$\bullet$~\normalfont\textbf]
\item To identify the performance of cognitive function in older people after neurofeedback training by assessments.
\item To identify the benificial effect of BCI neurofeedback training on fall risks reduction in older people by assessments.
\end{description}

\section{Study Design and Methods of Data Collection and Data Analysis}
\begin{comment}
Aim: To describe the study design. To clearly describe the data collection methods and outline the roles involved in data collection. To clearly describe the data analysis methods.
\end{comment}

The study will be a two-armed, randomized controlled trial. Subjects will be randomly allocated to two groups: Waitlist Group A, and Intervention Group B. A complete trial for each group will be 8 weeks. The trial flow chart is shown in figure \ref{fig:studyflow1}.

\begin{figure}[!ht]
	\includegraphics[scale=0.4]{studyflow1}
	\centering
    \caption{Flow chart of phases of parallel randomized controlled trial}
    \label{fig:studyflow1}
\end{figure}

Subjects-related procedures are stated below:


\textbf{Intervention Group B:}
In the first week, each participant will do a pre-training assessment(the  tools are elaborated in the below paragraph) for the study to recognize the their features including attention, gait or balance, and cognitive ability. The group will undergo a 24 sessions span over 8 weeks, where each week each participant will receive 3 sessions of training trial. The group stop receiving intervention by the end of week 8. The same assessments as the pre-trial will take place at the end of the 8 weeks, and 16 weeks.

\textbf{Waitlist Group A:}
In the first week, each participant in the group will also do a pre-trial assessment. Then the group will not receive any intervention until week 9. The group has the same pattern of training as group B from week 9 to week 16. The assessment will be then taken in week 9 before their first intervention, and the end of week 16.

\textbf{Data Collection and Processing:} The equipment used for the training  will include an commercial EEG headset and a computer screen. The EEG signals record will be made simultaneously during the training. To process data, the study will examine EEG profile to establish patterns underlies attention.  The participant will have to use concentration to manoeuvre or perform a series of tasks on the computer screen.Specifically, we will perform noise reduction to the collected signals by applying filters(IIR filter) to extract frequency domain details. A bandpass filter will be used to identify the EEG components of collected SMR for target frequency band. Real-time visual feedback will be rewarded on the screen during the training game \cite{Kober_2015}.

\textbf{Assessments}: To evaluate the effects of neurofeedback method, we plan to assess gait profile and neuropsychological profile using clinical screening tools. Gait and balance will be measured by Berg Balance Test, while their balance of confidence will be measured by CONFbal Scale. Their Visuo-spatial short term working memory will be measured by Corsi Block Tapping Test(CBTT). Execution and attention will be measured by the subtest within the Test of Attentional Performance(TAP). The cognitive profile will be measure by Mini-Mental State Examination (MMSE).

\begin{comment}
Gait and balance will be measured by Berg Balance Test, while their balance of confidence will be measured by CONFbal Scale. Frailty will be measured by Rockwood Frailty Score. Fall risk will be assessed by FRAT. Performance of daily living activity will be measured by Barthel. The cognitive profile will be measure by Mini-Mental State Examination (MMSE)\cite{Folstein_1975}. \textcolor{red}{(to be determined)}
\end{comment}


\begin{comment}
RBANS takes ~30 minutes to finish, which could be tiring to take for older people. \cite{Nouchi_2012} has some cognitive measurements
\end{comment}
\subsection{Study Setting}
Community/care home dwelling older adults.
\section{Sample and Recruitment}
\subsection{Eligibility Criteria}
\subsubsection{Inclusion criteria}
An intake assessment will be done at the pre- screen and the participant should fulfill the following criteria:

1. English-speaking adult of any gender over the age of 55+ staying in community. 

2. Non-urgent fallers in the fall group

3. Able to give written consent

4. Able to understand info sheet

5. Able to travel to the training site independently


 
\subsubsection{Exclusion criteria}
The following criterias will be excluded:

1. Adults diagnosed with dementia, cerebrovascular or neurological pathology.

2. Adults without capacity to consent as participants for the study.

3. Those who have primarily presented with a condition that is not within the remit of general surgeons.

\subsection{Sampling}
Participant assessment data sampling will be using database owned by NHS Mersey care.
EEG data sampling will be performed any time while every participant is in the training. The sampling data will be secured in an encrypted laptop by the University of Liverpool.

\subsubsection{Size of sample}
Estimate an effect size of 0.5 on the cognitive function improvement, as it indicates moderate to large improvement according to Cohen. We select power of 0.8, and significance level alpha of 0.05. Giving an allowance for up to 15\% drop out and lose rate, the approximate sample size for the recruitment is 39. The lose estimation was based on other similar studies that conducted trials related to cognitive function enhancement using neurofeedback training. For a 50-50 split, a group size will be 19, the other would be 20. 

\subsubsection{Sampling technique}
Participants will be recruited to the study by Mersey Care Fall Service team by the provision of an information sheet and a short presentation on the study. Potential participants will be assessed for eligibility and added to an fall group admission list. They will understand the information sheet and leave the contact details to the study team if the screening results meet the inclusive criteria. Once the BCI system is ready, the selected participants will be contacted by the study team and given further information over the email. Those wishing to proceed will have appointment made to the experiment sessions. Subjects are required to abstain from caffeinated beverage or other psychoactive
substances for 5 hours before a training session. A written consent sheet will be signed by each participant prior to the start of the first experiment sessions and will be stored in an encrypted space. 

\subsection{Recruitment and Sample Identification}
Subjects will be recruited from the fall group pre-determined by the occupational therapist from Mersey Care Fall Service team.

\subsubsection{Consent}
Potential participants who are interested in finding out more about the study will be given a participant information sheet. This includes details of the study including the purpose of the study and method of data collection and details of how confidentiality will be ensured. They will be given the opportunity to ask for clarification and further questions via email and telephone.

Capacity of consent will be done by exploring with potential participants their understanding of the purpose and nature of the research, the potential benefits (which will likely be to others rather than them as an individual) and risks. This will include how the data will be handled and confidentiality ensured. Individuals deemed not competent to consent will not proceed to the the training session.

A statement of consent will be given to each voluntary participant before attending the first training session. Once the participants understand and sign, the consent sheet will be stored.

\section{Ethical and Regulatory Considerations}

\subsection{Assessment and management of risk}
After the training, subjects may have improved their cognitive functions and seen a decrease in fall risks. Since the subjects are fall-prone, the risk of falling exist. To mitigate the risk, the study group put more eye on the participants during training.

Subjects may feel discouraged or sleepy during the session. It is also possible that the participants cannot endure the time length of each training. To prevent such issue, pre-trial pilot study will be conducted to ensure that the training games are designed and modified to a suitable specification for use in the trial.
\subsection{Research Ethics Committee (REC) and other Regulatory Review \& Reports}

\subsection{Regulatory Review \& Compliance}

\textbf{Regulatory Review \& Compliance}

\textbf{Amendments}

\subsection{Peer Review}
\subsection{Patient \& Public Involvement}
\subsection{Protocol Compliance}

\subsection{Data Protection and Patient Confidentiality}

\subsection{Indemnity}
\subsection{Access to the Final Study Dataset}

\subsection{Dissemination Policy}
\subsubsection{Dissemination policy}
\subsubsection{Authorship Eligibility Guidelines and any Intended Use of Professional Writers}

\printbibliography


\end{document}

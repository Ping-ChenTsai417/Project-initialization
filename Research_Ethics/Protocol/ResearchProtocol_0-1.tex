\documentclass{article}


\title{Study Protocol}
\author{Ping-Chen Tsai  \thanks{Supervisor: Dr. Lanaky Heba}}
\date{07 January 2021}
\usepackage[utf8]{inputenc}
\usepackage{xcolor}
\usepackage{xurl}
\usepackage{geometry}
\usepackage{microtype} %solve badbox
\usepackage[style=ieee, backend=biber, url=false, doi=false, isbn=false]{biblatex}
\usepackage{float}
\usepackage{comment}
\usepackage{enumitem} % for bullet points

\usepackage{color}   %May be necessary if you want to color links
\usepackage{hyperref}
\hypersetup{
    colorlinks,
    citecolor=blue,
    filecolor=black,
    linkcolor=black,
    urlcolor=black
}

\usepackage{graphicx}
\graphicspath{ {./image/} }
\addbibresource{ResearchProtocol.bib}

\setlength{\parskip}{1em}
\geometry{a4paper,scale=0.8}

\begin{document}

\maketitle

\tableofcontents
\pagebreak


\section{Background}
\begin{comment}
Aim: To place the study in the context of available evidence.
The background should be supported by appropriate references to published literature on the area of interest:
•	A thorough literature review of relevant studies and analysis, new research should build on formal review of prior evidence.
•	A brief description of the proposed study.
•	A description of the population to be studied.
•	relevant data from previous clinical trials such as efficacy, safety, tolerability

It should be written so it is easy to read and understand by someone with a basic sense of the topic who may not necessarily be an expert in the area. Some explanation of terms and concepts is likely to be beneficial. 
\end{comment}

Falls is one of the cause major causes to serious injuries in older people. About one-third of the frail adults age over 65 will experience at least one fall a year when living in their own homes, and the fall rate is three times higher in older people who are residents of care homes \cite{NHSFalls_2018, Robertson_2013}. Identifying and reducing risk of falls in the elderly hence have been extensively studied to carry out appropriate therapeutic interventions for the fallers. Risk assessment tools are used for falls screening and falls management. A wide range of fall risk assessment tools are used to identify fall risks of older people who are dwelling in the community \cite{PreventionofFalls2011} or in care homes \cite{Jung_2014, Kikkert_2017}. 

Ageing and mental disorders lead to gait and cognitive impairments. Correlation between cognitive impairments and fall risks have been identified in existing studies. Both literature evidence was provided and trials of experiments were conducted to in the studies to prove that older people with cognitive impairment are entitled to higher fall risk \cite{Yogev_Seligmann_2007,Martin_2012,Laurence_2017, Zhang_2019}. These studies mainly focused on examining damaged or declined cognitive functions impacts on gait in older people, and their relations to falls. Since falling in older people happened mostly when gait and balance control is poor, risk factors related to gait variables were investigated \cite{Zhang_2019, Mirelman_2012, Borowicz_2016}. Results from the studies ascertained that aspects of execution function including attention, processing speed and working memory in the cognitive domain are essential to control gait and postural tasks for human body. An appropriate intervention to improve cognitive function in older people therefore becomes a possibility to reduce the fall risk. 


\textcolor{red}{(To be determined where to put the following paragraphs:)}

A widely used therapy for improving cognitive function is electroencephalogram (EEG) based neurofeedback(NF) training. EEG based NF is a supportive training where the participant modulates his/her EEG activity based on real-time visual or auditory feedback from real world. Since EEG signals has been proved to expose one's intent and cognitive performance, expected EEG patterns transmitted from a healthy elderly could be the reference for a NF training. With repetitive volitional practice, reinforcement and sensory feedback,the participants will learn to better control neural activity that imposes positive effects on cognitive functions and gait or balance control \cite{Miladinovic_2020}. 



The last decade has seen the rapid ascent of non-invasive brain-computer-interface(BCI) technology for NF training. BCI provides communication between human brain and an external device. One randomized controlled study has showed the safety, usability, acceptability and efficacy of BCI intervention in older people \cite{Lee_2013}. Executive function, processing speed \cite{Nouchi_2012} and attention \cite{Arvaneh_2019} enhancement using EEG based BCI NF training was proved to be effective in a number of studies. Non-invasive BCI intervention requires slighter and less-expensive implementation compare to conventional fall-prevention programmes such as physical therapy and reconstruction of living environment(To be added). 

\section{Rationale}
% motivation: why we need to conduct the study? Why is the study reasonable?
\begin{comment}
Aim: To explain why the research questions/aim(s) being addressed are important and why closely related questions are not being covered. 
This should include:
•	A clear explanation of the research question/aim(s) and the justification of the study i.e. why the question is worth asking and, through consultation with public and patient groups, why this is worthwhile to participants or wider service delivery.
•	A contextual framing of the research question/aim(s) in relation to relevant policy and historical and/or literature bases.
\end{comment}

Falls in older people are major health care issue worldwide. Suitable interventions to reduce fall risk in older people with cognitive impairments were addressed in several studies. To date, emerging studies on the use of Brain-Computer-Interface(BCI) for cognitive training suggested that BCI could strengthen neural plasticity by neurofeedback training. Specifically, neurofeedback training applies BCI to change cognitive process and provide real-time feedbacks in either visual or acoustic form. The aim of the study is to investigate the performance of BCI approach on reducing fall risks in older adults.

\section{Theoretical Framework}


\section{Research Question/Aim}


\section{STUDY DESIGN and METHODS of DATA COLLECTION AND DATA ANALYIS}
\textbf{METHOD}: While NF training tasks are usually monotonous and repetitive, training tasks in the form of gaming are usually preferred. Game-like features in cognitive training can provide an intuitive rule to engage the participant which gives them self-efficacy \cite{McGonigal_2011}. This is to prevent the older participants from feeling demotivated or sleeping in the training session. 


\printbibliography


\end{document}

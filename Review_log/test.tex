\documentclass{article}


\title{Literature Review List}
\author{Ping-Chen Tsai  \thanks{Supervisor: Dr. Lanaky Heba. Blue colored indicates copied sentences/words.}}
\date{07 January 2021}
\usepackage[utf8]{inputenc}
\usepackage{xcolor}
\usepackage{xurl}
\usepackage{geometry}
\usepackage[style=ieee, backend=biber, url=false, doi=false, isbn=false]{biblatex}
\usepackage{float}
\usepackage{comment}
\usepackage{enumitem} % for bullet points

\usepackage{color}   %May be necessary if you want to color links
\usepackage{hyperref}
\hypersetup{
    colorlinks,
    citecolor=black,
    filecolor=black,
    linkcolor=black,
    urlcolor=black
}

\usepackage{graphicx}
\graphicspath{ {./image/} }
\addbibresource{bibliography.bib}

\setlength{\parskip}{1em}
\geometry{a4paper,scale=0.8}

\begin{document}

\maketitle

\tableofcontents
\pagebreak


\section{Fall risk assessment}
Keyword: fall prevention

\subsection{Article: Inaccurate judgement of reach is associated with slow reaction time, poor balance, impaired executive function and predicts prospective falls in older people with cognitive impairment}

This study integrates the assessment tool RJE with other neuropsychological function test to see fall risk of older people with CI.

Reach judgement error(RJE) measures the disparity between perceived and actual maximal reach(PR and MR) on elderly people whose cognitive function is impaired. The papers proposed RJE as a new fall risk assessment measure.The relationship between RJE and cognitive impairment of older people (CI) is studied as well. 
Absolute RJE = $(\mid  MR-PR\mid \div MR)\times 100$

\textcolor{blue}{Selection criteria for participants (Prospective cohort study): age=82 $\pm$ 7 years, female=52\% , people with mild-moderate CI
(MMSE 11$-$23; Addenbrooke's Cognitive Examination-Revised (ACE-R) $<$ 83). Participants were divided into tertiles based on their absolute RJE. detailed neuropsychological functions were assessed by existing tests to measure \textbf{memory, verbal fluency and visuospatial ability, sense of balance etc}. Medical history, demographic characteristics, usual level of mobility Falls were recorded prospectively over 12 months with the assistance of carers.}

\textbf{PR}: Ask the patient to judge furthest distant they can reach by their arm without practice.A pole started at a 120cm distance long and moved closer 5 cm increment. Patient was asked at every increment'Do you think you can safely reach it?' \textbf{MR}:The patient reached the pole at their chosen distance.

Larger absolute RJE meanse increased risk of prospective falls.Individuals with larger absolute RJE had poorer global cognition and executive function, increased concern about falling and poorer physical function (MR, reaction time, balance, Physiological Profile Assessment (\textbf{PPA}) fall risk scores)

Participants from the current study with CI were more likely to overestimate their reach ability (25\% vs 15\%), less likely to underestimate their reach ability (58\% vs 68\%) and equally likely to be accurate (17\%) 
when compared to the cognitively intact cohort \cite{Taylor_2018}.

By: Taylor, Morag E.; Butler, Annie A.; Lord, Stephen R.; Delbaere, Kim; Kurrle, Susan E.; Mikolaizak, A. Stefanie; Close, Jacqueline C.T. Experimental Gerontology. Dec2018, Vol. 114, p50-56. 7p. DOI: 10.1016/j.exger.2018.10.020.

\subsection{timed go-and-up test assessment tool}

Predictive Analytics Toolset for Health and Care Applications (PATH) (Hayn et al 2018), a MATLAB® based system. In addition to statistics, PATH was used for data management, signal processing, and Machine Learning functionalities (Sams et al 2019). \cite{Ziegl_2020}

Popular evaluation of balance and gait tools:
Get Up and Go Test; Timed Up and Go Test, the Berg Balance Scale, and the Performance‐Oriented Mobility Assessment\cite{PreventionofFalls2011}.


\subsection{Inclusion of medication-related fall risk in fall risk assessment tool in geriatric care units}
% \begingroup\small\noindent\textbf{Inclusion of medication-related fall risk in fall risk assessment tool in geriatric care units}\\
% \vspace{.3cm}
Medicine is a risk factor to the existing fall risk assessment tool used in geriatric care units.\textcolor{blue}{In psychotropics, the fall risk can be reduced by prospective management of adverse effects such as drowsiness, dizziness, slow reaction time and orthostatic hypotension.} The author proposed to build up a preliminary categorization of fall-risk increasing drugs as a fall risk assessment tool for geriatric care unit. Method used: The research takes records of drug use of 188 fall-experienced(at least once) patients, observe the adverse drug effects occurs on the patient (The side effects are connected to fall risk according to Summary of Product Characteristics, abriev. \emph{SmPC}), and categorized the patients in to \emph{high, medium and low} fall risk profile .


\textcolor{blue}{Cohort group selection criteria:
1)age$>$60, 2) evidence of at least one fall during their stay in a geriatric care unit, 3) evidence of fall documented by health care professionals (nurses, physicians)\cite{Michalcova_2020}.} 
\begin{flushleft}

By: Michalcova, Jana; Vasut, Karel; Airaksinen, Marja; Bielakova, Katarina. In: BMC Geriatrics. 20(1); BioMed Central Language: English, Database: Springer Nature Journals \par
DOI: 10.1186/s12877-020-01845-9

\end{flushleft}

\subsection{A fall prevention guideline for older adults living in long-term care facilities}


The paper provide a work-oriented guideline. It defines target population, conducts systematic literature review, develops \& evaluates a draft nursing intervention guideline(evaluation based on Scottish Intercollegiate Guidelines Network (SIGN)) and algorithm for older adult in LTC Long-term care.The resulting \textcolor{blue}{guideline consists of three-step assessment and three-step intervention approach.}

\textbf{Consequences of fall:}Among elderly fall population, residing at home, dwelling in nursing home and inpatients older adults are the majority. Physically, fall results in death or immobility; immobility leads to secondary health problems. Economically, medical costs arise. 

\textbf{Why fall risk assessment \& other fall prevention methods are needed?} The methods are superficial, one-off, meaning that \textcolor{blue}{risk factors are inspected in a cursory manner.} LTC are interested in assessing potential fall hazards and needs to intervene for hazard prevention. Heavy workloads and staff shortage also keeps systematic and comprehensive assessment away from LTC usually.

\textbf{Risk assessment guideline for factor of cognitive function} Evidence-based recommendations\emph{with literature evidence in parenthesis}. Senile dementia \emph{(Retrospective observational study, descriptive study)}, confusion\emph{(literature review)} and severe cognitive disability \emph{(Clustered randomized controlled trial, randomized controlled trial)} are mentioned. Intervention are strongly recommended. 

Check the paper for fall prevention algorithm, flowchart\cite{Jung_2014}.

By: Jung, D.; Kim, H.; Shin, S.. International Nursing Review, 1 December 2014, 61(4):525-533 Language: English. Blackwell Publishing Ltd 
DOI: 10.1111/inr.12131 , Database: Scopus®, pubMed


\section{BCI Cognitive functions}

Among different types of electroencephalography signals, \textbf{P300}, \textbf{steady state visual evoked potentials} and \textbf{motor imagery signals} were the most common. 

multiple sclerosis (MS) causes cognitive deficits \cite{Argento_2019}

\subsection{Using brain-computer interfaces: a scoping review of studies employing social research methods}

Game product to improve neurological condition?[12,13,14]


\subsection{BCI for stroke rehabilitation: motor and beyond}

BCI is a subtype of computer-assisted cognitive rehabilitation. Emotion-related deficits receive the least amount of attention for post-stroke rehabilitation. 
BCI is good for: motor function restoration \par
\textbf{How BCI do rehabilitation in general?}
2. real-time decoding brain dynamics 
3. decode patient's intention to move their limbs
4. provide a contigent sensory-motor feedback in forms of physical movement and visual feedback, etc.\par

BCI is able to bridge the stroke-induced gap between motor intention and sensory feedback.

\begin{figure}[!ht]
	\includegraphics[scale=0.9]{BCI rehabilitation for stroke flow chart}
	\centering
    \caption{BCI rehabilitation for stroke flow chart}
    
    \label{fig:BCI cog1}
\end{figure}
    

In figure \ref{fig:BCI cog1}, a good cycle of BCI rehabilitation is shown. Ut indicates that stroke may decline cognition functions. By BCI, neural links would be recovered and functional capabilities are restored. Good mood(happiness, not depressed or anxious) elevated motivation in cognitive rehabilitation. \textbf{Thus study group should have no depression history.} \par

\subsubsection{BCI intervention}

1. Neural feedback training:volitional/concious modulation of brain activations by patients. \textbf{A experimental trial ask the patient to look at the continuous visualisation of the brain activity from a specific region and are asked to volitionally up or down regulate this activity.}In details, up-regulate cortical activity in mu and beta bands in leisoned motor areas. Another experiment example is to ask patient to increase their attention index.

2. Operant conditioning: \textbf{Reward} the patient in visual/sensory feedback upon \textbf{successful elicitation on attempted action}. The reward is to use robot to move the disease-affected limb with target action imagined by the patient.\textbf{Punish} patient by no or negative feedback on insufficient motor imaginary activation. Such brain stimulus help drive neuroplastic changes.

3. Reinforcement of neural circuit by repetitive engagement: Train the patient to perform motor movement/imagination with tasks. The trial procoess is to record the complete brain activation pattern associated with specified tasks. Then repetitively identify hence reward the patient if he/she generate successful elicitation which match the recorded patterns. This is BCI-based, task-focused training using repetitive recruitment of cognitive circuit.

4. Hebian learning: Following principle of 'Hebbian plasticity' that Neurons that fire together wire together,  uses BCI to re-establish\textbf{(but how?...)} the contingency between cortical activity related to the imagined movement and actual movement(the feedback) 
Although patient presents loss of motor control, performing motor imagery can generate cortical activations associated with motor imagery \cite{Mane_2020}. 

By:Ravikiran Mane et al 2020 J. Neural Eng. 17 041001
Published 7 August 2020 • © 2020 The Author(s). Published by IOP Publishing Ltd. IOPScience.
Journal of Neural Engineering, Volume 17, Number 4

\subsection{A P300-Based Brain-Computer Interface for Improving Attention}

\textcolor{blue}{Moreover, BCI has been suggested as a potential neurofeedback training tool for improving cognitive performance (Lim et al., 2012; van Erp et al., 2012; Tih-Shih et al., 2013; Yang et al., 2018).}

\textcolor{blue}{Generally, P300 is a \textbf{positive deflection} in the EEG signal that appears approximately 300 ms after the presentation of an attended stimulus (Sutton et al., 1965)}

\subsubsection{P300 wave}

Figure \ref{fig:oddball paradigm} shows odd paradigm: a target(deviant) stimulus interrupts the repetitive background stimuli. P300 occurs \cite{van_Dinteren_2014}. 

\begin{figure}[!ht]
	\includegraphics[scale=0.55]{Schematic-overview-of-the-oddball-paradigm-and-an-ERP}
	\centering
    \caption{schematic review of the oddball paradigm and an event-related potential event-related potential (ERP)\cite{van_Dinteren_2014}}
    \label{fig:oddball paradigm}
\end{figure}

P300 amplitudes differed significantly across the lifespan. Amplitude of P300 is larger at:1) Older age. 2) \textcolor{red}{frontal than parietal area.} \textbf{Compensation-Related Utilization of Neural Circuits Hypothesis (CRUNCH)} is a model in relation to neurocognitive aging(or model of cognitive compensation
\cite{Reuter_Lorenz_2008}), which posits that the brain regions of older people tend to be more activated than younger people during performance of task.  

\textbf{P3a:} originates from stimulus-driven \textcolor{red}{frontal} attention mechanisms during task processing

\textbf{P3b:} originates from \textcolor{red}{temporal-parietal} activity associated with attention and related to subsequent memory processing.

\textbf{Hemispheric asymmetry in older adults:} Regions of overactivation are detected in older adults' prefrontal sites relative to younger adults. Overactivation is usually the  mirrored active sites in younger adults but in the opposite hemisphere. Overactivation in seniors brings more cognitive load that is beyond their processing capability \cite{Reuter_Lorenz_2008}. 

\subsubsection{Working memory}
Working memory is short-term memory that hold information temporarily. If too much information come together at once, it causes extensive cognitive load.

Intrinsic cognitive load: Make interconnection among completely different pieces of information in memory storage.

Extrinsic cognitive load: Quantity of items held in working memory.

Germane: Ability to process the information to make sense of it.

Working memory model consists of three main components:
\textbf{1)	Visuospatial sketchpad: for visual information
2)	Central executive: control and regulate cognitive process. Make working memory and long-term memory work together.
3)	Phonological loop: for verbal information, auditory memory}

Visuo-spatial sketchpad does not interfere with the short term process of the phonological loop. The subsystem of visuo-spatial sketchpad is spatial and object memory(or visual cache)

Addition to the working memory model: Episodic buffer have links to long-term memory, working memory( visuo-spatial sketchpad and phonological loop), and central executive. It is a buffer being able to bind together information from different sources to form integrated chunks \cite{Baddeley_2011}.

\begin{comment}
\subsection{Brain-Computer Interface (BCI): Types, Processing Perspectives and Applications}

\cite{Panoulas_2010}
\end{comment}
\subsection{Brain computer interface for communication and control}
keyword:
Electroencephalography,
augmentative 
communication,rehabilitation, neuroprothesis

Correlation between EEG signals and mental tasks been proved in the late 90s.

\textbf{Electrophysiological Recording:}Ion transport accross membrane. Transmembrane ion concentration is developed and thus electrical potential differences occurs on cells. The process of measuring electrical potential and current flows inside, outside and accross the cell membrane is called electrophysiological recording. \par

- Epidural(minimally invasive) and subdural(very invasive) electrode provide EEG with high topographical resolution. Intracortical(very very invasive) electrodes follows individual neurons.

Signal features from EEG: (e.g. amplitude of evoked potentials or sensory motor cortex rhythms, firing rate of cortical neurons \cite{WOLPAW_2002})


\section{Cognitive function and Risk of falls relationship}
keyword: fall risk,

https://courses.lumenlearning.com/boundless-ap/chapter/motor-pathways/

A motor system consists of the pyramidal and extrapyramidal systems. These two systems are the pathways 

Basal ganglia is a neural substrate connecting to the cerebral cortex(located at the base of the forebrain). It is in charge of voluntary motor control, in other words, the movement intention, habits, and cognitive functions. Most discussed neurocognitive disorders of basal ganglia: Parkinson's disease and Huntington's disease. When facing several possible impending movements or behaviours, basal ganglia takes part in deciding which to execute. The decision making process is called action selection. The neural signals from prefrontal cortex, where closely associated to executive functions, influences basal ganglia performing action selection. After behaviour decision is made, basal ganglia releases inhibition to permit a motor system to be active or idle. In summary, basal ganglia plays a key role in directing human movement with intent.

%\begin{comment}
\subsection{Cognitive and motor functioning in patients with multiple sclerosis: Neuropsychological predictors of walking speed and falls}
Poorer verbal memory increases risk of fall according to a binary logistic regression.Specific cognitive impairments are limiting the patients' mobility with \textbf{multiple Sclerosis} . Gait and balance evaluation along with cognitive assessment is suggested for patients with \textbf{multiple Sclerosis} \cite{D_Orio_2012}
\subsection{Sensorimotor, Cognitive, and Affective Functions Contribute to the Prediction of Falls in Old Age and Neurologic Disorders: An Observational Study}

Is cognitive impairment provide additional clues to sensorimotor deficits among VARIOUS population? Age range: $74.0 \pm 9.4$

The participants have been diagnosed cognitive impaired in different ways. Their sensorimotor function, cognitive function, and moodplus concern were assessed with existing assessment methods(by asking questions) and questionnaires. The participants were observed for 6-12 months.

As a result, \textcolor{blue}{Deficits in cognition/executive function and, depressive symptoms and concern about falling, are as significant as sensorimotor function for fall prediction.}

\cite{van_Schooten_2020}.
%\end{comment}
\subsection{Sensorimotor function and dizziness in neck pain: implications for assessment and management}

\begin{comment}
\begin{description}[font=$\bullet$~\normalfont\textbf]
\item [Bananas] yellow and banana shaped
\item [Apples] red and round
\item [Oranges] orange and round
\item [Lemons] yellow, kinda round
\end{description}
\end{comment}

\begin{description}[font=$\bullet$~\normalfont\textbf]
\item [Motor-control properties:]
motor control theories: 1)reflex theory(1906, Sherrington) 2)System Model. Goal-directed \& Task-oriented. 
\item [Systems involves in motor control]: 1) sensor: somatosensory , actuator: motor cortex 2)sensor: visual, actuator: Basal Ganglia.3)sensor: (keep balance), actuators: Cerebellum \& 	Central Pattern generators. 
\item [Fall-related concerns] Old people fear of falls may affect self-efficacy(or self intention). These elderly exhibit much poorer performance in balance tests \cite{Maki_1991} (\textcolor{blue}{blindfolded spontaneous-sway tests})
\end{description}

\textcolor{blue}{as people age, their sensorimotor systems deteriorate and their movement ability declines. The rate of decline can vary greatly depending on lifestyle, social and genetic factors. When muscle strength and somatosensory input decrease and reaction times increase, the risks of falls and developing fall-related concerns (FrC) increase.}

Somatosensory decline is a factor of fear of fall. Sensorimotor system is in charge of human postural control, which correlate to FrC. 
\subsection{The Fall in Older Adults: Physical and Cognitive Problems}
\cite{Laurence_2017}

\pagebreak

\section{EEG Primer Notes}

\textbf{synaptic potential}: most important source of EEG.

\textbf{neuron depolarization}: Typical neuron has a resting potential accross membrane. The interior cell is negatively charged relative to the outside. When the membrane potential becomes less negative(more positive

\textbf{cortex and thalamus relations}: 
Damage to a portion of the thalamus is associated with risk of coma. Damage in a portion of the thalamus can lead to sensory changes in a body part. Damage here can also cause movement disorders, lack of movement (motor disturbances).

Frequency band of EEG: $\delta$ (0.1-3 Hz), $\theta$ (4-7 Hz), $\alpha$ (8-13 Hz), $\beta$ (14-29 Hz),$\gamma$ (30-40 Hz).
\section{Review paper guide}

\textbf{An example of search strategy}: 

‘The literature search focused on risk factors and the prevention of falls among older adults living in LTC facilities as well as interventions. To this end, we categorized literature items into the following: individual research, review papers, systematic literature review and existing guidelines.’

‘Cumulative Indexing Nursing and Allied Health Literature (CINHAL), PubMed, SciVerse and Scopus were used as databases for the literature outside of South Korea. The Korean Education and Research Information Service (KERIS), Korean Institute of Science and Technology Information (KISTI), and the Korean Medical Database (Kmbase) were used for the domestic literature search. Studies published between January 2006 and October 2011 were searched using keywords such as ‘falls and nursing home’, ‘falls and long‐term care facilities’, ‘fall risk factor and old age’, ‘fall prevention and older people’, ‘fall intervention and older people’, ‘systematic review of falls and older people’ and ‘meta‐analysis of falls and older adults’.’ \cite{Jung_2014}

Guide of writing review paper \cite{reviewPaper_2001}.

Software tools for  meta-analysis: Meta-Discs
A few review-writing resources are listed below: \par
https://guides.lib.unc.edu/
systematic-reviews/PRISMA

\begin{description}[font=$\bullet$~\normalfont\textbf]
\item [Cochrane Handbook:] Chapter 15: Interpreting results and drawing conclusions
\item [PRISMA]: Flow of information. Preferred Reporting Items for Systematic Reviews and Meta-Analysis
\item [JBI Manual for Evidence Synthesis - Chapter 12.3 The systematic review] 
\end{description}
https://guides.lib.unc.edu/systematic-reviews/PRISMA

Investigate effective modalities to simulate the organization 
of brain funcional networks and manipulate neural activities by 
signal processing. The study manipulated activities by BCI 
could be modulated volitionally which is subject to neuralfeedback training 
strategy.

\printbibliography

\end{document}

